\documentclass[10pt]{beamer} 
%设置为 Beamer 文档类型,设置字体为 10pt,长宽比为16:9,数学字体为 serif 风格
\usepackage[UTF8]{ctex}
\usepackage{float}
% Useful packages
\usepackage{braket}
\usepackage{color}
\usepackage{amsmath}
\usepackage{graphicx}
%Information to be included in the title page:
\title[About Beamer] %optional
{Electrodynamics Lecture 2}
\subtitle{mathmatical knowledge and skills (continued)}
\author % (optional, for multiple authors)
{Yuxuan Zhang }
\institute[VFU] % (optional)
{
  School of Physics \quad
  Zhejiang University
}

\date[VLC 2021] % (optional)
{May 13th 2023}

\begin{document}
\frame{\titlepage}

\begin{frame}
    \frametitle{目录}
    \tableofcontents
\end{frame}

\section{积分计算回顾(一)}
\begin{frame}
    \frametitle{积分计算回顾(一)}
    \begin{itemize}
        \item 二重积分 $ds = r d\theta dr$\\
        拆成两个定积分,怎么拆?先计算一个小条带上的函数值乘以面积的和,在变化条带的位置,得到整个面积的。
        \item 三重积分 $dV = rdrd\theta dz = r^2 sin\theta d\theta d\phi dr$
        拆成三个定积分,先得到一个沿着轴移动的平面,平面上是二重积分,在对轴的变量进行积分。
    \end{itemize}
\end{frame}

\begin{frame}
    \frametitle{积分计算回顾(一)}
    \begin{itemize}
       \item  积分元变换的雅克比行列式的推导 \quad 用于不同曲线坐标的标量积分元 $ds,dV$\\
        矢量积分元——$\vec{dl},\vec{ds}$  在曲线坐标上三个方向的分量怎么写?     
        \item 第一类曲线积分,怎么寻找?法向量$\vec{n}$垂直于平面内任意曲线在这一点的切线方向,可以用两个特殊的曲线(与$x=x_0$或$y=y_0$的两条交线)叉乘得到法向方向。
    \end{itemize}
\end{frame}

\section{Kronecker $\delta$ 和 levi-civita 符号}
\begin{frame}
    \frametitle{ Kronecker $\delta$ 和 levi-civita 符号}
    \begin{itemize}
        \item 爱因斯坦求和规则 \quad $\delta_{ij}$对指标的缩并
        \item 排列的奇偶性 应用到$\epsilon_{ijk}$的“全反对称”性上
        \item 一个易错点:单项式内同一个指标最多出现两次。\quad 注意检查等式左右两边剩余的指标的一致性
        \item $\epsilon_{ijk} \epsilon_{ilm} = \delta_{jl}\delta_{km} - \delta_{jm}\delta_{kl}$ \quad 
        \item 应用 $\vec{A} \cdot \vec{B} $ \quad $(\vec{A}\times \vec{B})_i$ \quad $\nabla \cdot (\nabla \times \vec{A}) \equiv 0$
        \item 进一步的应用  $\star \star \quad \nabla \times (\nabla \times \vec{A}) = \nabla (\nabla\cdot \vec{A}) - \nabla^2 \vec{A}$ \\
注意这里定义了矢量的拉普拉斯$\Delta \vec{A} = \nabla^2 \vec{A}$
    \end{itemize}
\end{frame}

\end{document}