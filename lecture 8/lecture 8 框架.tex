\documentclass[10pt]{beamer} 
%设置为 Beamer 文档类型,设置字体为 10pt,长宽比为16:9,数学字体为 serif 风格
\usepackage[UTF8]{ctex}
\usepackage{float}
% Useful packages
\usepackage{braket}
\usepackage{color}
\usepackage{amsmath}
\usepackage{graphicx}
%Information to be included in the title page:
\title[About Beamer] %optional
{Electrodynamics Lecture 8}
\subtitle{ Static magnetic field (3)  }
\author % (optional, for multiple authors)
{Yuxuan Zhang }
\institute[VFU] % (optional)
{
  School of Physics \quad
  Zhejiang University
}

\date[VLC 2021] % (optional)
{June 8th 2023}

\begin{document}
\frame{\titlepage}

\begin{frame}
    \frametitle{目录}
    \tableofcontents
\end{frame}


\section{电矩、磁矩的进一步讨论}
\begin{frame}
    \frametitle{电矩、磁矩的进一步讨论}
    \begin{itemize}
    \item 磁偶极矩的定义 \quad 特殊情况 $\vec{m} = I \int d \vec{s} $
    \item 反过来用体电荷的推导来验证磁场力矩的公式,这其中有一个公式需要用$\int\partial_j (x_ix_lJ_j)d^3x = 0 $,得到$\int (J_ix_l +J_lx_i) d^3x = 0$   
    \item 磁矩的受力,因为非均匀的磁场导致,$\vec{F} = \nabla (\vec{m} \cdot \vec{B}) $延伸到定义磁场的能量
    \end{itemize}

\end{frame}



\end{document}
