\documentclass[10pt]{beamer} 
%设置为 Beamer 文档类型,设置字体为 10pt,长宽比为16:9,数学字体为 serif 风格
\usepackage[UTF8]{ctex}
\usepackage{float}
% Useful packages
\usepackage{braket}
\usepackage{color}
\usepackage{amsmath}
\usepackage{graphicx}
%Information to be included in the title page:
\title[About Beamer] %optional
{Electrodynamics Lecture 3}
\subtitle{mathmatical knowledge and skills (continued)}
\author % (optional, for multiple authors)
{Yuxuan Zhang }
\institute[VFU] % (optional)
{
  School of Physics \quad
  Zhejiang University
}

\date[VLC 2021] % (optional)
{May 16th 2023}

\begin{document}
\frame{\titlepage}

\begin{frame}
    \frametitle{目录}
    \tableofcontents
\end{frame}

\section{$\epsilon\; $和$\; \delta$ 的公式}
\begin{frame}
    \frametitle{$\epsilon\; $和$\; \delta$ 的公式}
    伪指标
    \begin{itemize}
    \item  $\epsilon_{imn} \epsilon_{jmn} = 2\delta_{ij}$ \quad $\epsilon_{ijk}\epsilon_{ijk} = 6$
    \item $\vec{A} \times (\vec{B} \times \vec{C} ) $
    \item $\nabla \times (\vec{A} \times \vec{B})$
    \item $\star \star \quad \nabla \times (\nabla \times \vec{A}) = \nabla (\nabla\cdot \vec{A}) - \nabla^2 \vec{A}$ 
    \end{itemize}

\end{frame}

\section{积分计算回顾(二)}
\begin{frame}
    \frametitle{积分计算回顾(二)}
    \begin{itemize}
    \item 第二类曲线积分——化为第一类曲线积分
    \item 第二类曲面积分——化为第一类曲面积分
    \end{itemize}
\end{frame}

\section{微分积分定理}
\begin{frame}
  \frametitle{积分微分定理}
  \begin{itemize}
      \item 高斯定理\quad 如果 $\nabla \cdot \vec{A} = 0$  第二类曲面积分 $\rightarrow$通量
      \item 斯托克斯定理 \quad 如果$\nabla \times \vec{A} = 0 $ \quad 第二类曲线积分
  \end{itemize}
\end{frame}

\section{$\delta \; $ function }
\begin{frame}
    \frametitle{$\delta \; $function}
    \begin{itemize}
        \item $\int \delta(x-a) f(x) dx = f(a)$
        \item 阶跃函数 $\int_{-\infty}^x dt \delta(t) = \theta(x) $
        \item $\delta(g(x)) = \Sigma_i \frac{1}{|g^\prime(x_i)|}\delta(x) $\\
        例:$\int_0^{+\infty} e^{-2x} \delta({cos x e^{-x}})dx = \frac{e^{\pi/2}}{e^\pi-1}$
        \item $\nabla \cdot \left( \frac{\vec{e_r}}{r^2}\right) = 4\pi \delta (\vec{0})$
    \end{itemize}
\end{frame}



\end{document}
