\documentclass[10pt]{beamer} 
%设置为 Beamer 文档类型,设置字体为 10pt,长宽比为16:9,数学字体为 serif 风格
\usepackage[UTF8]{ctex}
\usepackage{float}
% Useful packages
\usepackage{braket}
\usepackage{color}
\usepackage{amsmath}
\usepackage{graphicx}
%Information to be included in the title page:
\title[About Beamer] %optional
{Electrodynamics Lecture 12}
\subtitle{ the electromagnetic wave }
\author % (optional, for multiple authors)
{Yuxuan Zhang }
\institute[VFU] % (optional)
{
  School of Physics \quad
  Zhejiang University
}

\date[VLC 2021] % (optional)
{June 22th 2023}

\begin{document}
\frame{\titlepage}

\begin{frame}
    \frametitle{目录}
    \tableofcontents
\end{frame}

\section{介质中的麦克斯韦方程组}
\begin{frame}
    \frametitle{介质中的麦克斯韦方程组}
    注意电流分成三种,分别是自由电流、极化电荷产生的电流、磁化电流
\end{frame}

\section{麦克斯韦方程的解之一——电磁波}
\begin{frame}
    \frametitle{电磁波的简单介绍}
    波动方程与真空中的电磁波\\
    主要性质:
    \begin{itemize}
      \item 电磁波是横波
      \item 电场分量与磁场分量垂直
      \item 电磁波得偏振性
      \item 电磁波中能量的传播
    \end{itemize}

\end{frame}


\end{document}
