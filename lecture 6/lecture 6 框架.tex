\documentclass[10pt]{beamer} 
%设置为 Beamer 文档类型,设置字体为 10pt,长宽比为16:9,数学字体为 serif 风格
\usepackage[UTF8]{ctex}
\usepackage{float}
% Useful packages
\usepackage{braket}
\usepackage{color}
\usepackage{amsmath}
\usepackage{graphicx}
%Information to be included in the title page:
\title[About Beamer] %optional
{Electrodynamics Lecture 6}
\subtitle{Static electric and magnetic field }
\author % (optional, for multiple authors)
{Yuxuan Zhang }
\institute[VFU] % (optional)
{
  School of Physics \quad
  Zhejiang University
}

\date[VLC 2021] % (optional)
{May 30th 2023}

\begin{document}
\frame{\titlepage}

\begin{frame}
    \frametitle{目录}
    \tableofcontents
\end{frame}

\section{上一次课程内容补充}
\begin{frame}
    \frametitle{上一次课程内容补充}
    \begin{itemize}
        \item 导体表面受到的压强,单位面积的电磁作用力
        \item 无限长带电线+无限长圆柱面导体的电像法
    \end{itemize}
\end{frame}

\section{分离变量法解拉普拉斯方程}
\begin{frame}
    \frametitle{分离变量法解拉普拉斯方程}
    直角坐标和球坐标举例,理解分离变量的核心
\end{frame}

\section{静电场的多级展开}
\begin{frame}
    \frametitle{静电场的多级展开}
    \begin{itemize}

    \item 多级展开的核心是严格地按照$r^\prime/r$的各个阶来展开\\
    \item 区分我们通常说的偶极子,和真正的多级展开中的偶极项\\
    \end{itemize}
\end{frame}

\section{电荷在磁场中的受力}
\begin{frame}
    \frametitle{电荷在磁场中的受力}
    电荷磁场中的受力满足洛伦兹力公式,可以推导到带电体密度和带电线密度受到的力\\
    带电粒子受力会进行螺线运动,用假设一个用来平衡电场力的速度的方法进行推导。
\end{frame}


\end{document}
