\documentclass[10pt]{beamer} 
%设置为 Beamer 文档类型,设置字体为 10pt,长宽比为16:9,数学字体为 serif 风格
\usepackage[UTF8]{ctex}
\usepackage{float}
% Useful packages
\usepackage{braket}
\usepackage{color}
\usepackage{amsmath}
\usepackage{graphicx}
%Information to be included in the title page:
\title[About Beamer] %optional
{Electrodynamics Lecture 4}
\subtitle{Static electric field (1) }
\author % (optional, for multiple authors)
{Yuxuan Zhang }
\institute[VFU] % (optional)
{
  School of Physics \quad
  Zhejiang University
}

\date[VLC 2021] % (optional)
{May 20th 2023}

\begin{document}
\frame{\titlepage}

\begin{frame}
    \frametitle{目录}
    \tableofcontents
\end{frame}

\section{张量的概述(一)}
\begin{frame}
    \frametitle{张量的概述(一)}
    坐标变换-参考系变换可以被看做等价的内容,坐标变换中物理量变换的规律是我们关注的内容。
    \begin{itemize}
        \item $dx^i$ \quad $\partial_i$ \quad  $g_{i,j}$
        \item 反函数组的偏导数 满足的公式 $dx^i = \frac{\partial x^i}{\partial q^j}dq^j \quad \frac{\partial q^j}{\partial x^i}dx^i = dq^j $
        \\ $\frac{\partial x^i}{\partial q^j} \frac{\partial q^j}{\partial x^l} =\frac{\partial x^i}{\partial x^l} =\delta^i_l$
        \item 利用上面的内容定义协变矢量,从$dx_i$开始,矢量由于逆变和协变分为两种,也有其他满足更加复杂的变化规律的物理量,即更加高阶的张量
    \end{itemize}
\end{frame}

\section{静电场高斯定理的应用}
\begin{frame}
    \frametitle{高斯定理的应用}
    几种情况:点电荷、无限长带电直线、带电球体、无限大带电平面等
\end{frame}

\section{静电场的能量}
\begin{frame}
    \frametitle{静电场的能量}
    \begin{itemize}

    \item 补充一下“宏观微观”的问题\\
    \item 点电荷系统的能量 两个公式$W = \Sigma_{i< j} \frac{q_iq_j} {4\pi \epsilon_0 r_{ij}} $ 
    $w = \Sigma_i \frac{1}{2} q_i U_i = \Sigma_{i} \frac{1}{2} q_i \left( \Sigma_{j\neq i} \frac{q_j}{4\pi \epsilon_0 r_{ij}} \right) $ 
    \item 连续系统的电荷的能量
    \end{itemize}
\end{frame}

\end{document}
