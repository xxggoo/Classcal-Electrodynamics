\documentclass[10pt]{beamer} 
%设置为 Beamer 文档类型,设置字体为 10pt,长宽比为16:9,数学字体为 serif 风格
\usepackage[UTF8]{ctex}
\usepackage{float}
% Useful packages
\usepackage{braket}
\usepackage{color}
\usepackage{amsmath}
\usepackage{graphicx}
%Information to be included in the title page:
\title[About Beamer] %optional
{Electrodynamics Lecture 11}
\subtitle{ the electromagnetic field and the electromagnetic wave }
\author % (optional, for multiple authors)
{Yuxuan Zhang }
\institute[VFU] % (optional)
{
  School of Physics \quad
  Zhejiang University
}

\date[VLC 2021] % (optional)
{June 20th 2023}

\begin{document}
\frame{\titlepage}

\begin{frame}
    \frametitle{目录}
    \tableofcontents
\end{frame}

\section{物质中的磁场}
\begin{frame}
    \frametitle{物质中的磁场}
    磁化强度矢量,磁场强度矢量\\
    磁化产生的面电流和体电流的公式\\
    磁场强度和电场强度的对应性,电位移矢量和磁感应强度的对应性
\end{frame}

\section{静磁场的能量}
\begin{frame}
    \frametitle{静磁场的能量}
    从无到有建立一个稳恒电流,需要克服反电动势做功,这部分功转换为能量储存在了静磁场中。
\end{frame}

\section{介质中的麦克斯韦方程组}
\begin{frame}
    \frametitle{介质中的麦克斯韦方程组}
    向原先的关于$\vec{E}\vec{B}$的4个方程中加入另外的两个变量$\vec{H},\vec{D}$,同时给出两个关系式
\end{frame}


\section{电磁场的能量和能流}
\begin{frame}
    \frametitle{电磁场的能量和能流}
    能量密度标量、能流密度矢量$\vec{s} = \vec{E}\times \vec{H}$
\end{frame}


\section{狭义相对论中的电磁部分}
\begin{frame}
    \frametitle{狭义相对论中的电磁部分}
    \begin{itemize}
    \item     标量: 时空间隔、固有时、电荷量、粒子的静止能量
    \item 矢量:位移、速度、动量、源(电荷密度、电流密度)、$\partial_\mu$
    \item 二阶张量:电磁场张量$F^{\mu\nu}$
    \end{itemize}
    关系式:洛伦兹规范、方程组的势形式、方程组的场张量形式
\end{frame}



\end{document}
