\documentclass[10pt]{beamer} 
%设置为 Beamer 文档类型,设置字体为 10pt,长宽比为16:9,数学字体为 serif 风格
\usepackage[UTF8]{ctex}
\usepackage{float}
% Useful packages
\usepackage{braket}
\usepackage{color}
\usepackage{amsmath}
\usepackage{graphicx}
%Information to be included in the title page:
\title[About Beamer] %optional
{Electrodynamics Lecture 9}
\subtitle{ static magnetic field(4) and the  road to the Maxwell equation }
\author % (optional, for multiple authors)
{Yuxuan Zhang }
\institute[VFU] % (optional)
{
  School of Physics \quad
  Zhejiang University
}

\date[VLC 2021] % (optional)
{June 15th 2023}

\begin{document}
\frame{\titlepage}

\begin{frame}
    \frametitle{目录}
    \tableofcontents
\end{frame}


\section{磁感应强度的计算}
\begin{frame}
    \frametitle{磁感应强度的计算}
    毕奥萨伐尔定律以及简单的应用
    \begin{itemize}
        \item 带电圆环圆心处、z轴上一点处
        \item 无限长直导线
        \item 上述两个问题的两个延伸
    \end{itemize}

    磁场的旋度——得到安培环路定理——定理的几种简单应用情况:
    \begin{itemize}
        \item 无限长直电流
        \item 无限大平板
        \item 无限长螺线管
    \end{itemize}
\end{frame}

\section{磁场的散度——磁矢势}
\begin{frame}
    \frametitle{磁场的散度——磁矢势}
    对磁矢势的简单介绍,更加深入的内容应该在介绍完规范变换之后展开
\end{frame}


\section{电磁感应}
\begin{frame}
    \frametitle{电磁感应}
    \begin{itemize}
    \item 电动势的定义 $d\epsilon  = \vec{f} \cdot d\vec{l}$
    \item 电动力学中的两种电动势产生方式,分别对应了洛伦兹力和感生电场力
    \end{itemize}
\end{frame}

\section{麦克斯韦方程组}
\begin{frame}
    \frametitle{麦克斯韦方程组}
    位移电流作为最后一块拼图\\
    势和场之间的关系,势满足规范对称性
\end{frame}


\section{静磁场的边界条件}
\begin{frame}
    \frametitle{静磁场的边界条件}
    \begin{itemize}
\item    $B_{1n} = B_{2n}$ 
\item    $B_{1\tau} - B_{2\tau} = \mu_0 K$
    \end{itemize}
\end{frame}
\end{document}
