\documentclass[10pt]{beamer} 
%设置为 Beamer 文档类型,设置字体为 10pt,长宽比为16:9,数学字体为 serif 风格
\usepackage[UTF8]{ctex}
\usepackage{float}
% Useful packages
\usepackage{braket}
\usepackage{color}
\usepackage{amsmath}
\usepackage{graphicx}
%Information to be included in the title page:
\title[About Beamer] %optional
{Electrodynamics Lecture 10}
\subtitle{ Maxwell equation (continued) and electromagnetic field in matter }
\author % (optional, for multiple authors)
{Yuxuan Zhang }
\institute[VFU] % (optional)
{
  School of Physics \quad
  Zhejiang University
}

\date[VLC 2021] % (optional)
{June 19th 2023}

\begin{document}
\frame{\titlepage}

\begin{frame}
    \frametitle{目录}
    \tableofcontents
\end{frame}


\section{麦克斯韦方程组的势形式}
\begin{frame}
    \frametitle{麦克斯韦方程组的势形式}
    在两种情况下使用不同的规范:
    \begin{itemize}
        \item 静电静磁问题中的库伦规范,顺便给出磁矢势的一般解,给出磁矢势的多级展开,联系之前讲过的磁矩
        \item 电磁场问题中的洛伦兹规范
    \end{itemize}
\end{frame}

\section{物质中的电场}
\begin{frame}
    \frametitle{物质中的电场}
    极化强度矢量、电位移矢量\\
    极化产生的面电流和体电流的公式\\
    为什么可以用单位体积的平均偶极矩来计算极化的电场的大小\\
    含有介质的电场能量的公式究竟增加了哪一部分的能量?
\end{frame}



\end{document}
